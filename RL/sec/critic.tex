\section{基于 critic 的 Q-learning 算法}
Q-learning 是一种无模型(model-free)的强化学习算法
属于基于值函数的方法。它通过学习一个 Q 值表来评估在特定状态下采取某个动作的长期期望回报,从而找到最优策略。
Q-learning 的学习目标是一个最优的动作价值函数 $Q*(s, a)$,表示在状态 $s$ 下执行动作 $a$ 后,遵循最优策略所能获得的最大累积奖励,
满足贝尔曼最优公式:
\begin{equation}
    Q*(s, a) = \mathbb{E} [r + \gamma \max_{a'} Q(s', a')|s,a],
\end{equation}
其中 $r$ 是即时奖励,$\gamma$ 是折扣因子,$s'$ 是下一个状态。

在状态和动作有限的情况下,Q-learning 可以通过迭代更新 Q 值表来找到最优策略,利用时序差分(TD)方法更新 Q 值:
\begin{equation}
    Q(s, a) \leftarrow (1-\alpha) Q(s, a) + \alpha (r + \gamma \max_{a'} Q(s', a')-Q(s, a))
\end{equation}
其中 $\alpha$ 是学习率,控制更新的步长。但存在维度灾难以及不具备泛化能力,DQN 利用神经网络拟合 Q 值函数很好的
解决了上述问题。

\subsection{DQN}
DQN 关键包括:
\begin{enumerate}
    \item \textbf{Q 函数} 状态 $s$ 和动作 $a$ 下的长期回报期望,用神经网络拟合得到。
    \item \textbf{目标网络 Target Network} 用于估计目标 Q 值,与 Q 函数共享参数(定期更新),用于减少样本方差,
    目标 Q 值为:
    \begin{equation}
        y = r + \gamma \max_{a'} Q_{\text{target}}(s', a').
    \end{equation}
    \item \textbf{经验回放 Experience Replay} 用于训练时从经验池中随机采样一批数据(优先回放重要性高的经验),提高样本利用率,减小过拟合。
\end{enumerate}

DQN 算法流程:
\begin{enumerate}
    \item 初始化 Q 函数和目标网络参数相同;
    \item 与环境交互,存储经验 $(s,a,r,s')$ 到回放缓冲区;
    \item 从回放缓冲区中随机采样一批数据 $(s,a,r,s')$,计算目标值 $y$;
    \item 使用 Q 函数拟合 $(s,a,y)$,更新 Q 函数参数;
    \item 定期更新目标网络参数。
\end{enumerate}

训练得到 Q 值后,更新策略为
\begin{equation}
    \pi(s) = \arg\max_{a} Q(s, a).
\end{equation}
当 $a$ 是连续时,可通过采样、对 $a$ 求导、修改网络架构等方式获得更新后的策略。
另外,为增加探索能力,可采用 $\epsilon-greedy$ 策略,以一定概率 $\epsilon$ 随机选择动作,其余概率选择最优动作。

DQN 的一些训练改进:
\begin{enumerate}
    \item \textbf{Double DQN} 解耦动作选择和目标 Q 值计算,缓解过度估计问题,即目标 Q 值计算如下:
    \begin{equation}
        y = r + \gamma Q_{\text{target}}(s', \arg\max_{a'} Q_{\pi}(s', a')).
    \end{equation}
    \item \textbf{Prioritized Experience Replay} 引入样本重要性,根据样本 TD 误差更新样本权重,提高样本利用率。
    \item \textbf{Dueling Network} 利用两个网络分离状态值和动作优势值,提高学习效率。
    \item \textbf{Noisy Network} 加入噪声,提高探索能力,具体为每次在和环境交互获取数据时,对每个参数添加随机噪声(如高斯噪声)以
    提高探索性,比 $\epsilon-greedy$ 更有效。
    \item \textbf{结合 MC 和 TD} 结合 Monte Carlo 和 TD 方法,利用 multi-step 策略,减小方差。
\end{enumerate}




\section{Actor-Critic 混合方法}
Actor-Critic 是结合了策略梯度(Policy Gradient) 和 值函数(Value Function) 的混合架构:
\begin{enumerate}
    \item Actor(策略网络):负责生成动作,直接优化策略函数 $\pi_{\theta}(a|s;\theta)$,使得策略函数的期望回报最大化;
    \item Critic(价值网络):评估状态或动作的价值,通过 $V(s;\phi)$ 或 $Q(s,a;\phi)$ 计算。
\end{enumerate}

在直接策略梯度算法中,方程~\ref{eq:actor gradient} 中
\begin{equation}
    G_t^n - b = \sum_{t'=t}^{T}\gamma^{t'-t}r_{t'}^n-b,
\end{equation}
其具有不稳定性,方差大,因此可利用价值网络估计的期望值进行修正,有
\begin{equation}
    \mathbb{E}(G_t^n) = Q^{\pi_{\theta}}(s_t^n,a_t^n),\quad \text{baseline} = V^{\pi_{\theta}}(s_t^n);
\end{equation}
另外,
\begin{align}
    &Q^{\pi_{\theta}}(s_t^n,a_t^n) - V^{\pi_{\theta}}(s_t^n) \\
    &=\mathbb{E}(r_t^n + \gamma V^{\pi_{\theta}}(s_{t+1}^n) - V^{\pi_{\theta}}(s_t^n))\\
    &= r_t^n + \gamma V^{\pi_{\theta}}(s_{t+1}^n) - V^{\pi_{\theta}}(s_t^n).
\end{align}
以上即是 advantage function,即在策略梯度算法中,用价值网络估计的期望回报与当前状态的价值对
蒙特卡洛方法的估计值进行修正,从而减少方差。

训练过程中,可通过正则化项使得策略网络的输出分布熵更大以增加探索性。
另外,对于动作连续的情景,可以采用 Pathwise derivative policy gradient 方法使用确定性策略优化,
具体的网络结果中,actor 网络的输入为状态,输出为动作,critic 网络的输入为 actor 网络的输出,
训练 actor 网络即是使得 critic 网络值最大化,类似于 GAN 的思想,具体训练流程如下:
\begin{enumerate}
    \item 初始化主网络 actor $\pi_{\theta}$ 和 critic $Q_{\phi}$ 和目标网络 actor $\pi_{\theta^-}$ 和 critic $Q_{\phi^-}$;
    \item 使用当前策略进行环境交互采样数据(可添加噪声增加探索性),存储到回放缓冲区;
    \item 最小化 TD 误差更新 critic 网络:
        \begin{equation}
            L(\phi) = \mathbb{E}\left[(r+\gamma Q_{\phi^-}(s',\pi_{\theta^-}(s')) - Q_{\phi}(s,a))^2\right].
        \end{equation}
    \item 沿 $Q_{\phi}$ 梯度方向更新 actor 网络:
        \begin{equation}
            L(\theta) = \mathbb{E}(Q_{\phi}(s,\pi_{\theta}(s))).
        \end{equation}
    \item 训练几步后更新目标网络。
\end{enumerate}
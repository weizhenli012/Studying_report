\section{基于 Actor 的直接策略梯度方法}
对于采样的轨迹 episode $\tau^{n}$,policy gradient 方法即是采用梯度上升方法最大化期望回报:
\begin{equation}
    \text{maximize }\mathbb{E}_{\tau\sim p(\tau|\theta)}[R(\tau)] = \sum_{\tau} R(\tau)p(\tau|\theta)
    %=\text{maximize }\frac{1}{N}\sum_{n=1}^{N}R(\tau^{n})
    =\overline{R}_\theta.
\end{equation}
其中 $\theta$ 是待优化的参数,直接计算 $\overline{R}_\theta$ 的梯度:
\begin{align}
    \nabla_\theta\overline{R}_\theta = \sum_{\tau}R(\tau)p(\tau|\theta)\frac{\nabla p(\tau|\theta)}{p(\tau|\theta)}
    = \frac{1}{N}\sum_{n=1}^{N} R(\tau^{n})\nabla\log (p(\tau^{n}|\theta)), 
\end{align}
由于
\begin{align}
    p(\tau|\theta) &= p(s_1)p(a_1|s_1,\theta)p(r_2,s_2|s_1,a_1)p(a_2|s_2,\theta)p(r_3,s_3|s_2,a_2) \cdots p(r_T,s_T|s_{T-1},a_{T-1})\\
    &=p(s_1)\prod_{t=1}^{T-1}p(a_t|s_t,\theta)p(r_{t+1},s_{t+1}|s_t,a_t),
\end{align}
因此
\begin{align}
    \nabla \log (p(\tau|\theta)) &= \sum_{t=1}^{T-1} \nabla \log(p(a_t|s_t,\theta)),\\
    \nabla \overline{R}_\theta &= \frac{1}{N}\sum_{n=1}^{N}\sum_{t=1}^{T-1}R(\tau^{n}) \nabla \log(p(a_t^n|s_t^n,\theta)).
\end{align}

这里梯度上升的直接理解就是若 $R(\tau^n)$ 为正,则增加 $p(a_t^n|s_t^n,\theta)$ 的概率,相反若为负则减小;另一方面,
即使 $R(\tau^n)$ 为正,但其中并不是每个行为策略都能导致 $R(\tau^n)$ 增加,因此有一些改进方法,如
对每个行为策略,计算该动作起到结束的奖励和作为 return,即
\begin{equation}\label{eq:actor gradient}
    \nabla \overline{R}_\theta = \frac{1}{N}\sum_{n=1}^{N}\sum_{t=1}^{T-1}(\sum_{t'=t}^{T}\gamma^{t'-t}r_{t'}^n-b) \nabla \log(p(a_t^n|s_t^n,\theta)).
\end{equation}
其中 $b$ 代表baseline,起到调整正负(策略相对好坏之分)、减小方差的作用,之后会引入优势函数(advantage function)。



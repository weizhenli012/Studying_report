\section{RL 基础介绍}
强化学习的核心是智能体(Agent)与环境(Environment)的交互学习框架,其基本要素包括:
\begin{enumerate}%[leftmargin=2.5cm]
  \item \textbf{状态(State)} 环境在时刻$t$的观测表示,记为$S_t \in \mathcal{S}$。例如:\\
  - 机器人坐标 $(x,y)$ \\
  - 棋盘局面的矩阵编码
  
  \item \textbf{动作(Action)} 智能体在状态$s$下的可执行操作,记为$A_t \in \mathcal{A}(s)$。例如:\\
  - 离散动作:围棋的落子位置 \{上, 下, 左, 右\} \\
  - 连续动作:机械臂关节扭矩 $[0, 1]^n$
  
  \item \textbf{奖励(Reward)} 环境对动作的即时反馈,记为$R_{t+1} \in \mathbb{R}$。例如:\\
  - 游戏得分变化:$+1$(得分),$-10$(触碰障碍) \\
  - 稀疏奖励:仅在任务完成时获得$+100$
\end{enumerate}

% ================================================
\subsection{马尔可夫决策过程(MDP)}
\textbf{马尔可夫性}:未来状态仅依赖当前状态与动作:
\[
P(S_{t+1}|S_t, A_t) = P(S_{t+1}|S_t, A_t, S_{t-1}, A_{t-1}, \dots).
\]
MDP五元组定义为:
\[
\langle \mathcal{S}, \mathcal{A}, P, R, \gamma \rangle:
\]
\begin{itemize}%[leftmargin=2.5cm]
  \item $\mathcal{S}$ 状态空间(有限或无限集合)
  \item $\mathcal{A}$ 动作空间(可能依赖状态$s$)
  \item $P(s'|s,a)$ 状态转移概率:$\sum_{s'} P(s'|s,a) = 1$
  \item $R(s,a,s')$ 奖励函数(可简化为$R(s,a)$或$R(s)$)
  \item $\gamma$ 折扣因子:$\gamma=0$只关注即时奖励,$\gamma \to 1$强调长期回报
\end{itemize}

强化学习的目标即是在此框架下最大化\textbf{累积折扣奖励}:
\begin{equation}
  G_t = \sum_{k=0}^{\infty} \gamma^k R_{t+k+1}
\end{equation}
优化策略$\pi(a|s)$以最大化期望回报$\mathbb{E}_\pi[G_t]$。

% ================================================
\subsection{价值函数与贝尔曼方程}
\subsubsection{状态价值函数(State-Value Function)}
策略$\pi$下状态$s$的长期价值,定义为从$s$出发的期望回报:
\[
V^\pi(s) = \mathbb{E}_\pi \left[ G_t \mid S_t = s \right]
\]
其中累积回报:
\[
G_t = \sum_{k=0}^\infty \gamma^k R_{t+k+1}
\]

\subsubsection{动作价值函数(Action-Value Function)}
在状态$s$采取动作$a$后遵循策略$\pi$的期望回报:
\[
Q^\pi(s,a) = \mathbb{E}_\pi \left[ G_t \mid S_t = s, A_t = a \right]
\]

\subsubsection{贝尔曼方程(Bellman Equation)}
价值函数的递归分解:
\[
V^\pi(s) = \sum_{a} \pi(a|s) \sum_{s'} P(s'|s,a) \left[ R(s,a,s') + \gamma V^\pi(s') \right]
\]
\[
Q^\pi(s,a) = \sum_{s'} P(s'|s,a) \left[ R(s,a,s') + \gamma \sum_{a'} \pi(a'|s') Q^\pi(s',a') \right]
\]

\subsubsection{最优价值函数}
存在最优策略$\pi^*$使得:
\[
V^*(s) = \max_\pi V^\pi(s), \quad Q^*(s,a) = \max_\pi Q^\pi(s,a)
\]
满足贝尔曼最优方程:
\[
V^*(s) = \max_{a} \sum_{s'} P(s'|s,a) \left[ R(s,a,s') + \gamma V^*(s') \right]
\]
\[
Q^*(s,a) = \sum_{s'} P(s'|s,a) \left[ R(s,a,s') + \gamma \max_{a'} Q^*(s',a') \right]
\]

% ================================================
\subsection{强化学习方法分类}
\begin{table}[htbp]
  \centering
  \caption{强化学习方法对比}
  \begin{tabular}{lcc}
    \toprule
    \textbf{方法类型} & \textbf{策略表示} & \textbf{更新目标} \\
    \midrule
    基于值函数(Critic) & 隐式(如$\epsilon$-greedy) & 逼近$Q^*(s,a)$ \\
    基于策略(Actor) & 显式$\pi_\theta(a|s)$ & 最大化$J(\theta) = \mathbb{E}[G_t]$ \\
    Actor-Critic & 显式策略 + 值函数 & 策略梯度+值函数引导 \\
    \bottomrule
  \end{tabular}
\end{table}

\subsubsection{基于值函数的方法}
通过优化值函数间接改进策略,例如:
\begin{itemize}
  \item 动态规划(DP):迭代计算$V^\pi(s)$
  \item 蒙特卡洛(MC):采样轨迹估计$Q^\pi(s,a)$
  \item 时序差分(TD):结合DP与MC,如Q-learning:
  \[
  Q(s,a) \leftarrow Q(s,a) + \alpha \left[ R + \gamma \max_{a'} Q(s',a') - Q(s,a) \right]
  \]
\end{itemize}

\subsubsection{基于策略的方法}
直接参数化策略并通过梯度上升优化:
\[
J(\theta) = \mathbb{E}_{\tau \sim \pi_\theta} \left[ \sum_{t=0}^\infty \gamma^t R_t \right]
\]
梯度计算公式:
\[
\nabla_\theta J(\theta) = \mathbb{E} \left[ \sum_{t=0}^T \nabla_\theta \log \pi_\theta(a_t|s_t) G_t \right]
\]

\subsubsection{Actor-Critic框架}
\begin{itemize}
  \item \textbf{Actor}:策略$\pi_\theta(a|s)$,通过策略梯度更新
  \item \textbf{Critic}:值函数$V_w(s)$或$Q_w(s,a)$,通过TD误差更新
  \item 优势函数(Advantage Function)减少方差:
  \[
  A(s,a) = Q(s,a) - V(s) \implies \nabla_\theta J(\theta) = \mathbb{E} \left[ \nabla_\theta \log \pi_\theta(a|s) A(s,a) \right]
  \]
\end{itemize}